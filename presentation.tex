\documentclass{beamer}
\usepackage{graphicx}
%Information to be included in the title page:
\title{Sample title}
\author{Anonymous}
\institute{Overleaf}
\date{2021}

\begin{document}

\frame{\titlepage}

\begin{frame}
\frametitle{Background 1-2 min}

\includegraphics[scale = .5]{lenstra.jpeg}

\begin{itemize}
\item Hendrik Lenstra Jr. recieved his doctorate from the University of Amsterdam in 1977.

\item Discovered Elliptic Curve Factorization (ECM) in 1987.

\item ECM is third-fastest known factoring algorithm and the best algorithm for finding divisors not exceeding 50-60 digits.

\item The largest factor found using ECM has 83 digits.
\end{itemize}
\end{frame}

\begin{frame}
\frametitle{Preliminaries 2 mins} 
\begin{itemize}
\item Let $E$ be an elliptic curve over $\mathbb{Z}/N\mathbb{Z}$ of the form
$$
	y^2 = x^3 + ax + 1
$$
such that $4a^3 + 27 \in \left(\mathbb{Z}/N\mathbb{Z}\right)^*$. This forces non singularity and ensures $P = (0,1)$ is on the curve.

\item Definition 6.3.1 (Power Smooth). Let $B$ be a positive integer. If $n$ is a positive integer with prime factorization 
$$
    n = \prod p_i^{e_i},
$$
then $n$ is $B$-power smooth if $p_i^{e_i} \leq B$ for all $i$. 

\item Example $30 = 2\cdot 3\cdot 5$ is $B$ power smooth for $B \geq 5$, but $150 = 2\cdot 3 \cdot 5^2$ is not $5$-power smooth.
\end{itemize}
\end{frame}

\begin{frame}
\frametitle{Motivation 1-2 mins}
% Explain why p-1 not being B power smooth for a fixed B is a problem
\begin{itemize}
\item Fix $B \in \mathbb{N}$. Let $p \in \mathbb{N}$ such that $p-1$ is not $B$- power smooth. 

\item Recall, in Pollard $p-1$, this would be equivalent to not having $p-1 \vert m = \text{lcm}(1,2, \ldots, B)$; i.e. $a^m \not \equiv 1 \pmod{p}$.

\item On the interval $[10^{15}, 10^{15} + 10000]$ 15 percent of the primes $p$ are such that $p-1$ is not $10^{6}$-power smooth. 

\item The idea of ECM is to replace modular exponentiation on $\left(\mathbb{Z}/N\mathbb{Z}\right)^*$ by repeated addition of points on $E\left(\left(\mathbb{Z}/N\mathbb{Z}\right)^*\right)$

\item Recall, by the Hasse-Weil bound we can reduce the size of our group by $2\cdot \sqrt{p}$. 
\end{itemize}
\end{frame}

\begin{frame}
\frametitle{Elliptic Curve Factorization 2 mins}
Algorithm 6.3.10 (Elliptic Curve Factorization Method). Let $N$ and $B$ be positive integers.
\begin{itemize}
\item[1.] Compute $m =$ lcm$(1,2,\ldots, B)$. 
\item[2.] Choose $a \in \mathbb{Z}/N\mathbb{Z}$ such that $4a^3 + 27 \in \left(\mathbb{Z}/N\mathbb{Z}\right)^*$. This forces $P = (0,1)$ to be a point on $y^2 = x^3 + ax +1$ over $\mathbb{Z}/N\mathbb{Z}$. 
\item[3.] Try to compute $mP$. If at somepoint we cannot compute a sum of points some denominator $g$ is not coprime to $N$, then $gcd(g,N)$ is a nontrivial divisor of $N$.
\end{itemize}
\end{frame}

\begin{frame}
\frametitle{Analogy to Pollard p-1 1 min}
\begin{table}[h!]
  \begin{center}
    \caption{Let $E$ be an elliptic curve, and $m = lcm(1,2,\ldots,B)$ for some $B$}
    \label{tab:table1}
    \begin{tabular}{|c|c|} % <-- Alignments: 1st column left, 2nd middle and 3rd right, with vertical lines in between
      \textbf{Pollard $p-1$} & \textbf{ECM} \\
      \hline
      $\mathbb{Z}/N\mathbb{Z}$ & $E\left( \mathbb{Z}/N\mathbb{Z} \right)$\\ &\\
      $g \in (\mathbb{Z}/N\mathbb{Z})^*$ & $(0,1)$ \\ & \\
      $g^m \equiv 1 \pmod{N}$ & $mP \notin E\left( \mathbb{Z}/N\mathbb{Z} \right)$ \\ &\\
      $gcd(g^m-1, N)$ & $gcd(m,N)$
    \end{tabular}
  \end{center}
\end{table}

\begin{itemize}
\item If Pollard $p-1$ fails, we have no choice but to increase $B$. 
\item However, ECM has a second option. We can choose another random elliptic curve. 
\end{itemize}
\end{frame}

\begin{frame}
\frametitle{Why it works 1-2 mins}

\end{frame}

\begin{frame}
\frametitle{Example by hand 2 mins}

\end{frame}

\begin{frame}
\frametitle{Implementation 2 mins}

\end{frame}

\begin{frame}
\frametitle{Run Time Analysis/Comparison 2 mins}

\end{frame}

\begin{frame}
\frametitle{Coded Example 2 mins}

\end{frame}

\begin{frame}
\frametitle{Animation 1 min}

\end{frame}

\end{document}